\documentclass[11pt]{article}

\usepackage[utf8]{inputenc}
\usepackage[margin=1in]{geometry}
\usepackage{amssymb}
\usepackage{graphicx}
\usepackage{hyperref}

\begin{document}
\begin{titlepage}
   \begin{center}
       \vspace*{1cm}

       \textbf{\huge Dumbo interpreter}

            
       \vspace{1.5cm}

       \textbf{HUYLENBROECK Florent\\
       			BOSSART Laurent}

       \vfill
            
       Devoir pour le cours de compilation\\
            
       \vspace{0.8cm}
            

       UMONS\\
       Annee academique 2019-2020
            
   \end{center}
\end{titlepage}
\newpage
\tableofcontents
\newpage
\section{Introduction}
Dans le cadre de notre cours de compilation, il nous a été demandé de créer le programme nommé \emph{"dumbo\_interpreter"}, permettant de générer facilement des fichiers contenant du texte. La génération se fait en fonction de données reçues en paramètres. Pour ce faire nous devions créer le langage \emph{dumbo}.\\
Ce programme prend en entrée 3 paramètres :
\begin{itemize}
\item Un fichier data, contenant du code dumbo servant à initialiser des variables.
\item Un fichier template, dans lequelle sera injecté les variables du fichier data, selon l'évaluation du code Dumbo contenu dans ce fichier template.
\item Un fichier output, dans lequel sera écrit le résultat de la génération.
\end{itemize}

\newpage
\section{Grammaire Dumbo}
En plus d'une grammaire de base donnée dans l'énoncé du devoir, certaines fonctionnalités additionnelles nous ont été demandées. Voici la grammaire complète qui a été implémentée :
\begin{center}
\begin{tabular}{|lcl|}
\hline
$<programme>$ 					& $\rightarrow$ 	& $<txt>$\\
								&					& $|<txt><programme>$\\
$<programme>$ 					& $\rightarrow$ 	& $<dumbo$\_$block>$\\
								&					& $|<dumbo$\_$block><programme>$\\
$<txt>$							& $\rightarrow$  	& $[a-zA-Z0-9;\&<>"\_ =-.\setminus / \setminus n\setminus s:,]+$\\
$<dumbo\_ block>$				& $\rightarrow$ 	& $\{\{<expression\_ list>\}\}$\\
								&					& $|$ $\{\{$   $\}\}$\\
$<expression\_ list>$ 			& $\rightarrow$     & $<expression>;<expression\_ list>$\\
								&					& $|$ $<expression>;$\\
$<expression>$					& $\rightarrow$ 	& $print$ $<string\_ expression>$\\
								&					& $|$ $for$ $<variable>$ $in$ $<string\_ list>$\\
								&					& $do$ $<expression\_ list>$ $endfor$\\
								&					& $|$ $for$ $<variable>$ $in$ $<variable>$\\
								&					& $do$ $<expression\_ list>$ $endfor$\\
								&					& $|$ $<variable>:=<string\_ expression>$\\
								&					& $|$ $<variable>:=<string\_ list>$\\
								&					& $|$ $if$ $<boolean\_ expression>$ $do$ $<expression\_ list>$ $endif$\\
$<string\_ expression>$ 		& $\rightarrow$ 	& $<string>$\\
								&					& $|$ $<comparable\_ expression>$\\
								&					& $|$ $<string\_ expression>.<string\_ expression>$\\
$<string\_ list>$ 				& $\rightarrow$		& $(<string\_ list\_ interior>)$\\
$<string\_ list\_ interior>$ 	& $\rightarrow$ 	& $<string>$\\
								&					& $|$ $<string>,<string\_ list\_ interior>$ \\
$<variable>$ 					& $\rightarrow$ 	& $[a-zA-Z0-9\_ ]$\\
$<string>$						& $\rightarrow$		& $'[a-zA-Z0-9;\&<>"\_ =-.\setminus / \setminus n\setminus s:,]+'$\\
$<comparable\_ expression>$ 	& $\rightarrow$		& $<variable>$\\
								&					& $|$ $<number\_ expression>$\\
$<number\_ expression>$ 		& $\rightarrow$ 	& $[0-9]+(.[0-9]+)?$\\
								&					& $|$ $<number\_ expression>[+-*/]<number\_ expression>$\\
								&					& $|$ $-<number\_ expression>$\\
$<boolean>$						& $\rightarrow$ 	& $true$\\
								&					& $|$ $false$\\
								&					& $|$ $<comparable\_ expression>$ $([<>=]|!=)$ $<comparable\_ expression>$\\
$<boolean\_ expression>$ 		& $\rightarrow$ 	& $<boolean>$\\
								&					& $|$ $<boolean>$ $and$ $<boolean\_ expression>$\\
								&					& $|$ $<boolean>$ $or$ $<boolean\_ expression>$\\
\hline
\end{tabular}
\end{center}
Certaines modifications ont été apportées à la grammaire de base :
\begin{itemize}
\item Une $<string\_ expression>$ peut maintenant comprendre une $<comparable\_ expression>$ à la place d'une $<variable>$. Cependant, une $<comparable\_ expression>$ peut contenir une $<variable>$.
\item Étant donné qu'il nous a été demandé de gérer la division, les nombres réels sont supportés, en extension aux nombres entiers.
\end{itemize}

\newpage
\section{Gestion des boucles \emph{for} et \emph{if}}
Une fois les instructions booléennes correctement implementées, la gestion des boucles \emph{if} ne nous a pas semblée compliquée. Nous l'avons gérée comme pour l'assignation, la concaténation, et le print : par la construction d'un tuple $("if",boolean\_ expression,instruction\_ list)$. Lors de l'interprétation du template, si un tel tuple est rencontré et que la condition est respectée, les instructions sont exécutées récursivement par notre fonction $apply\_ function$ laquelle prend un tuple en entrée et applique l'opération définie par ce tuple.\\[.5cm]
Pour la boucle \emph{for}, il a été nécessaire d'ajouter une fonctionnalité à notre interpréteur Dumbo : la gestion de la \emph{profondeur}. Nous avons donc redéfini la structure de données permettant de stocker nos variables : d'un dictionnaire \textbf{variables}[nom] = \emph{valeur}, nous sommes passés à un dictionnaire de dictionnaire \textbf{variables}[profondeur][nom] = \emph{valeur}.
Ainsi, à chaque entrée dans une boucle for, la profondeur est incrémentée d'un (valeur de départ = $0$). Les valeurs de la $<string\_ list>$ sur laquelle on itère sont copiées à cette profondeur-là. Le corps de la boucle est exécuté récursivement comme pour nos autres fonctions. Ensuite, à la sortie de cette boucle, le sous-dictionnaire à ce niveau de profondeur est effacé et la profondeur est décrémentée d'un.\\[.5cm]
Cette modification nous a amené à modifier plusieurs autres fonctions dans notre code. Par exemple, la fonction qui identifie une variable à partir d'une chaine de caractères a du être adaptée pour rechercher la variable en partant du niveau de profondeur actuel jusqu'au niveau 0, s'arrêtant à la première correspondance. L'assignation se fait maintenant de la même manière : si une variable correspondante est trouvée, une nouvelle valeur lui est assignée, sinon une variable est créée au niveau de profondeur actuel.\\[.5cm]
Cette modification nous permet de faire tourner plusieurs boucles \emph{for} imbriquées, lesquelles peuvent utiliser le même nom de variable dans leurs corps respectifs.

\newpage
\section{Difficultés rencontrées}
Une difficulté évidente a été de ne pas pouvoir se concerter en binôme dans la vraie vie, à cause du confinement. Mais ceci sort du cadre de cette section du rapport.\\
La première difficulté rencontrée a été de gérer l'appel de fonctions. Nous avons trouvé la solution dans la ressource qui nous a été fournie pendant les séances de travaux pratiques sur l'analyse lexicale : \url{http://www.dabeaz.com/ply/ply.html} (C'est d'ailleurs dans cette documentation que nous avons trouvé la majorité des réponses à nos problèmes). Nous avons donc suivi cette idée de créer un tuple dont le premier élement est le nom de la fonction et les suivants sont les paramètres nécessaires à l'exécution de cette fonction. Ainsi, l'exécution récursive des fonctions devenait possible afin de résoudre d'abord les arguments avant de calculer la valeur de la fonction.\\
La seconde difficulté rencontrée a été la gestion du \emph{for}. Nous avons essayé plusieurs alternatives avant d'en arriver à utiliser notre système de profondeur. Nous avons d'abord commencé par séparer nos variables en deux listes : \textbf{variables} et 
\textbf{variables\_loop} mais cette solution, bien qu'au final fonctionelle pour la gestion d'une seule boucle \emph{for}, ne nous convenait pas.\\

\section{Conclusion}
Ce projet était intéressant dans le cadre de notre cours de compilation. En plus d'approfondir notre compréhension du fonctionnement d'un compilateur, il nous a permis de découvrir un type de programmation que nous n'avions pas rencontré avant. Nous sommes satisfaits des solutions que nous avons mises en place afin de résoudre les problèmes demandés.

\end{document}